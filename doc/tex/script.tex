%TCIDATA{LaTeXparent=0,0,THESIS.TEX}
                      

\chapter{Script Files\label{chapter:script}}

There are only $4$ commands used in the script files, namely:

\begin{enumerate}
\item  \textbf{\$TARGET}

\item  \textbf{\$PLATFORM}

\item  \textbf{\$RADAR}

\item  \textbf{\$SIMULATION}
\end{enumerate}

After each of these commands, a series of parameters need to be given. They
can be separated by spaces or commas. The number of parameters depend on the
parameters themselves and can vary considerably. This approach has been
taken to provide flexibility if the descriptions are complex, and to provide
better readability if the descriptions are simple. Comments can be made with
an ``!'' (exclamation mark); all text behind it will be ignored until the
next line break. The parameters follow the input dialogs closely, so if
there should be any doubts it might be helpful to look at the corresponding
dialog.

The order in which the objects are defined in a script file is important.
They should be in the following order: platforms, targets, radars and then
simulations.

\section{User-defined functions}

The flowcharts for all $4$ functions will be shown. Some of them contain a
block called ``Array Definition.'' These blocks contain data on user-defined
functions and the flow diagram is shown below in Figure~\ref{userdef}. Up to
three functions can be defined in one block, for example $x$-, $y$- and $z$%
-coordinates versus time. For each function an interpolation method needs to
be specified (cubic, matched-filter or linear).

\FRAME{fhFU}{3.8233in}{1.5714in}{0pt}{\Qcb{Flow diagram for user-defined
function block}}{\Qlb{userdef}}{userdef.wmf}{\special{language "Scientific
Word";type "GRAPHIC";maintain-aspect-ratio TRUE;display "ICON";valid_file
"F";width 3.8233in;height 1.5714in;depth 0pt;original-width
366pt;original-height 149.125pt;cropleft "0";croptop "1";cropright
"1";cropbottom "0";filename
'J:/SARSIM2/DOC/TEX/GRAPHICS/Userdef.wmf';file-properties "XNPEU";}}

User-defined functions can be defined either inline or via an external data
file. Inline means that the actual values for the function(s) are included
in the parameter list of the command (compact solution). This is useful if
there are only a small number of samples. With a larger number of samples it
is more practical to use external data files. In this case it is only
necessary to specify the filename. The files should be in the same directory
as the script file. Note that the filename string should not contain commas
or spaces. The structure of the data (either inline or in a file) depends on
the functions to be defined but are of the following pattern:

%TCIMACRO{
%\TeXButton{\tiny}{\tiny
%}}
%BeginExpansion
\tiny
%
%EndExpansion
\begin{verbatim}
[Number of samples for function 1], [X1], [Y1], [X2], [Y2], [X3], [Y3] ....[Xn][Yn]
[Number of samples for function 2], [X1], [Y1], [X2], [Y2], [X3], [Y3] ....[Xn][Yn]
[Number of samples for function 3], [X1], [Y1], [X2], [Y2], [X3], [Y3] ....[Xn][Yn]
\end{verbatim}

%TCIMACRO{\TeXButton{\normalsize}{\normalsize}}
%BeginExpansion
\normalsize%
%EndExpansion

First the number of samples is given, then the $x$- and $y$-coordinates for
all the samples are given, then it starts again with the next function (if
there is one). For example specifying the path of a platform needs $3$
functions, one for each coordinate versus time. An example of an ``INLINE''
user-defined trajectory function is given below (the comments are not
necessary):

%TCIMACRO{
%\TeXButton{\tiny}{\tiny
%}}
%BeginExpansion
\tiny
%
%EndExpansion
\begin{verbatim}
INLINE
2 ! Number of samples for Position X (m) vs Time (s)
-1, -100 1, 100
1 ! Number of samples for Position Y (m) vs Time (s)
0, 10000
1 ! Number of samples for Position Z (m) vs Time (s)
0, 2000
\end{verbatim}

%TCIMACRO{\TeXButton{\normalsize}{\normalsize}}
%BeginExpansion
\normalsize%
%EndExpansion

The equivalent in a ``FILE'' would look identical (ignoring the comments):

%TCIMACRO{
%\TeXButton{\tiny}{\tiny
%}}
%BeginExpansion
\tiny
%
%EndExpansion
\begin{verbatim}
! Data file for : Position X (m) vs Time (s), Position Y (m) vs Time (s), Position Z (m) vs Time (s)
! Structure : [No. Samples] [Time (s) 1] [Position X (m) 1] [Time (s) 2] [Position X (m) 2] ...
! [No. Samples] [Time (s) 1] [Position Y (m) 1] [Time (s) 2] [Position Y (m) 2] ...
! [No. Samples] [Time (s) 1] [Position Z (m) 1] [Time (s) 2] [Position Z (m) 2] ...
2
-1 -100, 1 100
1
0 10000
1
0 2000
\end{verbatim}

%TCIMACRO{\TeXButton{\normalsize}{\normalsize}}
%BeginExpansion
\normalsize%
%EndExpansion

From the given samples the program will interpolate the requested values as
shown in Figure~\ref{interpolation}.

\section{The \$TARGET command}

A flowchart of the ``\$TARGET'' command is shown in Figure~\ref{target}.

\FRAME{fhFU}{6.557in}{7.0301in}{0pt}{\Qcb{Flow diagram of the \$TARGET
command}}{\Qlb{target}}{target.wmf}{\special{language "Scientific Word";type
"GRAPHIC";maintain-aspect-ratio TRUE;display "ICON";valid_file "F";width
6.557in;height 7.0301in;depth 0pt;original-width 472.125pt;original-height
506.3125pt;cropleft "0";croptop "1";cropright "1";cropbottom "0";filename
'D:/SARSIM2/DOC/TEX/GRAPHICS/Target.wmf';file-properties "XNPEU";}}

\section{The \$PLATFORM command}

A flowchart of the ``\$PLATFORM'' command is shown in Figure~\ref{platform}.

\FRAME{fhFU}{4.8179in}{7.4677in}{0pt}{\Qcb{Flow diagram of the \$PLATFORM
command}}{\Qlb{platform}}{platform.wmf}{\special{language "Scientific
Word";type "GRAPHIC";maintain-aspect-ratio TRUE;display "ICON";valid_file
"F";width 4.8179in;height 7.4677in;depth 0pt;original-width
0pt;original-height 0pt;cropleft "0";croptop "1";cropright "1";cropbottom
"0";filename 'D:/SARSIM2/DOC/TEX/GRAPHICS/Platform.wmf';file-properties
"XNPEU";}}

\section{The \$RADAR command}

The flowcharts of the ``\$RADAR'' command are shown in Figure~\ref{radar1}
and in Figure~\ref{radar2} (Split up due to size considerations).

\FRAME{fhFU}{4.1269in}{8.6697in}{0pt}{\Qcb{Flow diagram of the \$RADAR
command (1st part)}}{\Qlb{radar1}}{radar1.wmf}{\special{language "Scientific
Word";type "GRAPHIC";maintain-aspect-ratio TRUE;display "ICON";valid_file
"F";width 4.1269in;height 8.6697in;depth 0pt;original-width
329.4375pt;original-height 694.1875pt;cropleft "0";croptop "1";cropright
"1";cropbottom "0";filename
'J:/SARSIM2/DOC/TEX/GRAPHICS/Radar1.wmf';file-properties "XNPEU";}}

\FRAME{fhFU}{5.4872in}{8.6135in}{0pt}{\Qcb{Flow diagram of the \$RADAR
command (2nd part)}}{\Qlb{radar2}}{radar2.wmf}{\special{language "Scientific
Word";type "GRAPHIC";maintain-aspect-ratio TRUE;display "ICON";valid_file
"F";width 5.4872in;height 8.6135in;depth 0pt;original-width
394.8125pt;original-height 620.75pt;cropleft "0";croptop "1";cropright
"1";cropbottom "0";filename
'J:/SARSIM2/DOC/TEX/GRAPHICS/Radar2.wmf';file-properties "XNPEU";}}

\section{The \$SIMULATION command}

A flowchart of the ``\$SIMULATION'' command is shown in Figure \ref{simulat}.

\FRAME{fhFU}{5.0047in}{6.0563in}{0pt}{\Qcb{Flow diagram of the \$SIMULATION
command}}{\Qlb{simulat}}{simulat.wmf}{\special{language "Scientific
Word";type "GRAPHIC";maintain-aspect-ratio TRUE;display "ICON";valid_file
"F";width 5.0047in;height 6.0563in;depth 0pt;original-width
0pt;original-height 0pt;cropleft "0";croptop "1";cropright "1";cropbottom
"0";filename 'D:/SARSIM2/DOC/TEX/GRAPHICS/Simulat.wmf';file-properties
"XNPEU";}}
