%TCIDATA{LaTeXparent=0,0,THESIS.TEX}
                      

\chapter{Command Summary\label{chapter:commands}}

This chapter explains all available commands.

\section{Main Window}

\subsection{Overview}

After starting Sarsim you will see a screen similar to the one shown in
Figure~\ref{mainscreen}. In this specific example a file has been loaded for
demonstration purposes. On the left side of the window, the current
simulation time, the current look-angle and a list of all objects is given.
The simulation time is useful in order to see how objects move with respect
to time. It can be changed by pressing the \textbf{Earlier} or \textbf{Later}
buttons, or by entering a specific time into the edit box. The look angle is
given by an azimuth and elevation angle, where the azimuth angle is measured
clockwise from the $y$-axis, and the elevation angle from the $x$-$y$ plane.
The object list shows all current objects, namely point targets (PTs),
platforms and radars. They are sorted with respect to platforms, and below
each platform an indented list of all objects relative to this platform is
given. Objects can be altered by clicking or double-clicking the respective
list item. On top of the screen a few speed buttons are placed. Clicking on
them will create a new object of the selected type.

\FRAME{fhFU}{4.9199in}{3.7369in}{0pt}{\Qcb{Main window}}{\Qlb{mainscreen}}{%
mainscre.gif}{\special{language "Scientific Word";type
"GRAPHIC";maintain-aspect-ratio TRUE;display "ICON";valid_file "F";width
4.9199in;height 3.7369in;depth 0pt;original-width 0pt;original-height
0pt;cropleft "0";croptop "1";cropright "1";cropbottom "0";filename
'D:/SARSIM2/DOC/TEX/GRAPHICS/Mainscre.gif';file-properties "XNPEU";}}

\subsection{Mouse Commands}

The mouse is used to rotate the coordinate system and to zoom in and out. It
is also used to modify or delete objects.

\begin{itemize}
\item  \textbf{ROTATION}: To rotate the axes, push and hold the left mouse
button and move the mouse in the required direction. Moving it up and down
will change the elevation angle, moving it left and right changes the
azimuth angle.

\item  \textbf{ZOOM}: To zoom in or out, push and hold the right mouse
button and move it up or down. The distance marks on the axes should start
moving.

\item  \textbf{MODIFY / DELETE}: Point targets, platforms and radars can be
modified by either selecting the required object on the list or
right-clicking the required object in the coordinate system. A menu
containing options to modify or delete the object will appear.
\end{itemize}

\subsection{Menu Commands}

\subsubsection{File}

\begin{itemize}
\item  \textbf{New:} Creates a new simulation. If the current simulation has
not been saved, the user will be asked if the current file should be saved.

\item  \textbf{Open:} Loads a script file from disk (default extension =
``.scr''). The structure of the script files will be explained in detail in
Chapter~\ref{chapter:script}.

\item  \textbf{Save:} Saves the current simulation file to disk. If no name
has been given yet, the user will be asked to enter a name.

\item  \textbf{SaveAs:} Saves the current file with a new name.

\item  \textbf{Exit:} Exits the program.
\end{itemize}

\subsubsection{Geometry}

\begin{itemize}
\item  \textbf{New Target:} Creates a new point target. A detailed
description is given in Section~\ref{newtarget}.

\item  \textbf{New Platform:} Creates a new platform. A detailed description
is given in Section~\ref{newplatform}.

\item  \textbf{New Radar:} Creates a new radar. A detailed description is
given in Section~\ref{newradar}.
\end{itemize}

\subsubsection{Simulation}

\begin{itemize}
\item  \textbf{Raw Return:} This option shows the received return signal
mixed down to baseband. A detailed description is given in Section~\ref
{simulationwindow}.

\item  \textbf{Range Compressed:} This options shows the range compressed
return, also mixed down to baseband. A detailed description is given in
Section~\ref{simulationwindow}.

\item  \textbf{Previously Stored:} The parameters of a simulation can be
stored so that they can be recalled quickly. This includes the window range,
sampling frequency, radar used, output file name and some more parameters. A
list of all stored simulations will be given. An option to delete a
simulation setup is also given.
\end{itemize}

\section{New Target Window}

\label{newtarget}Creating a new target brings up a dialog window shown in
Figure~\ref{newtargetdialog}.

\FRAME{fhFU}{4.395in}{3.154in}{0pt}{\Qcb{New target dialog}}{\Qlb{%
newtargetdialog}}{newtarge.gif}{\special{language "Scientific Word";type
"GRAPHIC";maintain-aspect-ratio TRUE;display "ICON";valid_file "F";width
4.395in;height 3.154in;depth 0pt;original-width 0pt;original-height
0pt;cropleft "0";croptop "1";cropright "1";cropbottom "0";filename
'D:/SARSIM2/DOC/TEX/GRAPHICS/Newtarge.gif';file-properties "XNPEU";}}The
following parameters can be configured:

\begin{itemize}
\item  \textbf{Platform name (up to 15 characters):} The platform to which
the current point target belongs. The coordinates entered will be relative
to the origin of this given platform.

\item  \textbf{Position }$x$\textbf{, }$y$\textbf{\ and }$z$: The
coordinates of the point target in meters.

\item  \textbf{Standard deviation of }$x$\textbf{, }$y$\textbf{\ and }$z$%
\textbf{\ position}: Introduce Gaussian distributed random jitter.

\item  \textbf{Radar Cross Section}: The radar cross section in square
meters.

\item  \textbf{Radar Cross Section deviation}: If the target scintillates,
the standard deviation of the RCS can be specified. The distribution is
Gaussian.

\item  \textbf{Isotropic reflectivity:} The incidence angle of the ray from
the radar is irrelevant, i.e.~the radar cross section will always be
constant.

\item  \textbf{Directional reflectivity}: This imitates a surface for which
the reflected energy is given as a function of the incidence angle (azimuth
and elevation angle). If the gain is set to ``Cos'', the radar cross section
changes exactly like the projected surface of a disc seen from a specific
angle, i.e.~effective RCS $=$ RCS $\cdot $ $\cos ($incidence angle$)$. No
reflection takes place on the backside. It is possible to define a specific
gain pattern by clicking on the ``Definition'' button. This is explained in
detail in Section~\ref{userfunc}.
\end{itemize}

\section{New Platform Window}

\label{newplatform}Creating a new platform brings up the dialog window shown
in Figure~\ref{newplatdialog}.

\FRAME{fhFU}{4.062in}{3.5034in}{0pt}{\Qcb{New platform dialog}}{\Qlb{%
newplatdialog}}{newplatf.gif}{\special{language "Scientific Word";type
"GRAPHIC";maintain-aspect-ratio TRUE;display "ICON";valid_file "F";width
4.062in;height 3.5034in;depth 0pt;original-width 0pt;original-height
0pt;cropleft "0";croptop "1";cropright "1";cropbottom "0";filename
'D:/SARSIM2/DOC/TEX/GRAPHICS/Newplatf.gif';file-properties "XNPEU";}}The
following parameters can be configured:

\begin{itemize}
\item  \textbf{Platform name}: The name of this platform (up to 15
characters) for future referencing.

\item  \textbf{Position, ``Stationary'' or ``Trajectory''}: If the position
of the platform is stationary, the $x$, $y$ and $z$ coordinates need to be
entered. This will place the platform at the given coordinate on the
``Earth'' platform. If the position of the platform is defined as a
trajectory, the position for every time instance can be defined. The given
points will be interpolated (parabolic, linear or low-pass filtered). This
will be explained in Section~\ref{userfunc}.

\item  $x$\textbf{, }$y$\textbf{\ and }$z$\textbf{-rotation}: This describes
the rotation angles of the platform around the $x$, $y$ and $z$ axes seen
relative to the ``Earth'' platform. Note that the platform can be
``aligned''.

\item  \textbf{Align }$x$\textbf{-axis to path}: If this box is provided
with a check mark, the coordinate system will be rotated in such a way that
the $x$-axis will coincide with the path at any given time instance. To
express it differently, the azimuth and elevation angle of any point on the
path will be parallel to the $x$-axis of the platform. The $y$-axis will
however still be parallel to the ``Earth'' platform. An example where this
is useful is if a platform is used to describe the motion of an aircraft,
then the nose of the aircraft will always point in the direction it is
flying. An example is give in Chapter~\ref{chapter:examples}.

\item  The parameters in the box ``Motion errors'' describe any deviations
from the ideal position or rotation. The distribution is Gaussian. There is
the option to define motion errors by giving an integration envelope which
will be convolved with Gaussian distributed noise, but this option is still
in development.
\end{itemize}

\section{New Radar Window}

\label{newradar}Creating a new radar brings up a dialog window shown in
Figure~\ref{newradardialog1}.

\FRAME{fhFU}{3.9946in}{3.9799in}{0pt}{\Qcb{New radar dialog - Page 1}}{\Qlb{%
newradardialog1}}{newradar1.gif}{\special{language "Scientific Word";type
"GRAPHIC";maintain-aspect-ratio TRUE;display "ICON";valid_file "F";width
3.9946in;height 3.9799in;depth 0pt;original-width 0pt;original-height
0pt;cropleft "0";croptop "1";cropright "1";cropbottom "0";filename
'D:/SARSIM2/DOC/TEX/GRAPHICS/Newradar1.gif';file-properties "XNPEU";}}

The following parameters can be configured on this page:

\begin{itemize}
\item  \textbf{Radar name}: For convenience there can be more than one radar
in the simulation. The radar name has to be specified for future reference.

\item  \textbf{Platform name}: The name of the platform onto which the radar
will be positioned. The radar is always positioned at the origin of the
platform.

\item  \textbf{Pulse type}: Choose either monochrome, chirp pulse or
user-defined pulses. If a chirp pulse is chosen, the chirp bandwidth needs
to be specified.

\item  \textbf{Pulsewidth}: The pulsewidth in nanoseconds.

\item  \textbf{Envelope}: The pulse can be multiplied by an envelope
function to model more realistic pulses as shown in Figure~\ref{pulse}.%
\FRAME{fhFU}{2.4578in}{1.3353in}{0pt}{\Qcb{Pulse envelope}}{\Qlb{pulse}}{%
pulse.eps}{\special{language "Scientific Word";type
"GRAPHIC";maintain-aspect-ratio TRUE;display "ICON";valid_file "F";width
2.4578in;height 1.3353in;depth 0pt;original-width 0pt;original-height
0pt;cropleft "0";croptop "1";cropright "1";cropbottom "0";filename
'D:/SARSIM2/DOC/TEX/GRAPHICS/Pulse.eps';file-properties "XNPEU";}}

\item  \textbf{Pulse Repetition Frequency}: The PRF can be defined either as
a constant or it can vary from pulse to pulse as shown in Figure~\ref{prf}.
The data wraps around, such that PRI$[x]=$PRIArray$[$modulus$(x,n)]$. Note
that all times are given in seconds.\FRAME{fhFU}{4.8023in}{1.2687in}{0pt}{%
\Qcb{User-defined PRFs}}{\Qlb{prf}}{prf.eps}{\special{language "Scientific
Word";type "GRAPHIC";maintain-aspect-ratio TRUE;display "ICON";valid_file
"F";width 4.8023in;height 1.2687in;depth 0pt;original-width
0pt;original-height 0pt;cropleft "0";croptop "1";cropright "1";cropbottom
"0";filename 'D:/SARSIM2/DOC/TEX/GRAPHICS/Prf.eps';file-properties "XNPEU";}}

\item  \textbf{Centre frequency}: Specify the centre frequency of each
pulse. It can be constant, stepped or completely arbitrary.

\item  \textbf{Power output}: Total power output of radar in kW.

\item  \textbf{System losses}: Total system losses in dB.

\item  \textbf{Noise temperature}: Noise temperature of system in Kelvin.
\end{itemize}

Noise is modelled in a simple way by specifying noise temperature of the
receiver. The following example will show how to convert between the noise
temperature and a required S/N ratio at a certain distance:

The noise power can be calculated by:

\begin{equation}
N=k\cdot T\cdot B
\end{equation}

where $k$ is Boltzmann's constant $(=1.38\cdot 10^{-23}$ Joules/degree$)$, $%
T $ is the equivalent receiver noise temperature in Kelvin and $B$ is the
bandwidth of the pulse (for chirp pulses $=$ Chirp bandwidth, for monochrome
pulses $=\frac{1}{\text{Pulsewidth}})$. Assuming we want to add the noise to
a given complex sample $S=I+j\cdot Q$, the values representing complex
voltage. The resulting vector (signal + noise) $SN=IN+j\cdot QN$ is then
calculate as follows: 
\begin{eqnarray}
IN &=&I+\func{GaussianNoise}\left( \sqrt{N}\right) \\
QN &=&Q+\func{GaussianNoise}\left( \sqrt{N}\right)
\end{eqnarray}
where $\func{GaussianNoise}()$ returns a random value with mean = 0 and
given standard deviation.

Consider a simple simulation with a point target (RCS = 1\thinspace m$^{2}$)
at a distance of $1$\thinspace km, and a radar with a power output of $1$%
\thinspace kW. The pulse should have monochrome modulation and have a width
of $1000$\thinspace ns. The noise temperature is set to $100$~Kelvin, the
centre frequency is $1$\thinspace GHz. The signal power received would be:

\begin{eqnarray}
P_{rx} &=&\frac{P_{tx}\cdot \lambda ^{2}\cdot \sigma }{(4\cdot \pi
)^{3}\cdot d^{4}} \\
&=&\frac{1000\cdot 0.3^{2}\cdot 1}{(4\cdot \pi )^{3}\cdot 1000^{4}} \\
&=&4.53\cdot 10^{-14}\,\text{W}
\end{eqnarray}

The noise power would be:

\begin{eqnarray}
N &=&k\cdot T\cdot B \\
&=&1.38\cdot 10^{-23}\cdot 100\cdot \frac{1}{10^{-6}} \\
&=&1.38\cdot 10^{-15}\,\text{W}
\end{eqnarray}

which gives a power S/N ratio of:

\begin{eqnarray}
\frac{S}{N} &=&\frac{4.53\cdot 10^{-14}\,\text{W}}{1.38\cdot 10^{-15}\,\text{%
W}} \\
&=&32.8 \\
&=&15.2\,\text{dB}
\end{eqnarray}

Converting to amplitudes (assuming a 1$\Omega $ resistor):

\begin{eqnarray}
V_{\text{received}} &=&\sqrt{P_{rx}} \\
&=&2.13\cdot 10^{-7}\,\text{V}
\end{eqnarray}

and

\begin{equation}
\text{Noise Voltage}=3.715\cdot 10^{-8}\,\text{V}
\end{equation}

The signal to noise ratio with respect to amplitudes would be $5.73$.

A second page of configurable parameters can be selected. It is shown in
Figure~\ref{newradardialog2}.

\FRAME{fhFU}{3.9799in}{3.9799in}{0pt}{\Qcb{New radar dialog - Page 2}}{\Qlb{%
newradardialog2}}{newradar2.gif}{\special{language "Scientific Word";type
"GRAPHIC";maintain-aspect-ratio TRUE;display "ICON";valid_file "F";width
3.9799in;height 3.9799in;depth 0pt;original-width 0pt;original-height
0pt;cropleft "0";croptop "1";cropright "1";cropbottom "0";filename
'D:/SARSIM2/DOC/TEX/GRAPHICS/Newradar2.gif';file-properties "XNPEU";}}The
following parameters can be configured on this page:

\begin{itemize}
\item  \textbf{Transmitter and receiver antenna gain}: Either isotropic, a
typical $\sin (x)/x$ pattern, or user-defined.

\item  \textbf{Antenna Direction}

\item  \textbf{Matched Filter Window}

\item  \textbf{Sensitivity Time Control}
\end{itemize}

The radar is positioned at the origin of the given platform.

\section{Simulation Window}

\subsection{\label{simulationwindow}Overview}

There are two processing options available:

\begin{itemize}
\item  \textbf{A raw return simulation}: In this case the transmitted signal
is not range-compressed (i.e.~convolved with a replica). The sampling rate
is automatically set such that all detail is shown at the current screen
resolution.

\item  \textbf{A range-compressed simulation}: The return gets
range-compressed. Because range-compression is computationally intensive,
the sampling rate can be set. Note that setting low sampling rates (less
than three times Nyquist rate) may result in unfamiliar looking graphs.
\end{itemize}

A typical simulation window is shown in Figure~\ref{simulationwin}.

\FRAME{fhFU}{4.478in}{3.6288in}{0pt}{\Qcb{A typical simulation window}}{\Qlb{%
simulationwin}}{rawretur.gif}{\special{language "Scientific Word";type
"GRAPHIC";maintain-aspect-ratio TRUE;display "ICON";valid_file "F";width
4.478in;height 3.6288in;depth 0pt;original-width 0pt;original-height
0pt;cropleft "0";croptop "1";cropright "1";cropbottom "0";filename
'D:/SARSIM2/DOC/TEX/GRAPHICS/Rawretur.gif';file-properties "XNPEU";}}

The simulation window shows the received return signal mixed down to
baseband. On the right hand side a window with slant range (in meters)
versus azimuth (in seconds) is shown. The maximum amplitude (in mV) is shown
at the bottom. This is only an approximation based on the current window. In
the top left corner the name of the selected radar is shown. The returned
signals have been calculated for this specific radar. Select a different
radar to display returns for a different radar.

To change the simulation window, edit the range and azimuth values in the
window box. It is possible to zoom in interactively by left-clicking and
dragging on the simulation window. If the simulation window is
right-clicked, it will zoom out by a factor of two in the slant range and
azimuth directions.

The ``Display'' box selects whether the real, imaginary part or the phase or
magnitude is displayed. The display of the pulses is colour-coded, where
blue represents positive, black zero and red negative values. In case of a
graphics card supporting $256$ or more colours, the colours will change
smoothly according to the amplitude. The contrast can be changed by moving
the slider. If one zooms in such that successive pulses are separated by
more than $10$ screen pixels in the azimuth direction, the pulses will not
be colour-coded anymore, but shown as proper graphs.

There is a important point which needs to be clarified: The simulation
window shows slant range in the $x$-direction and azimuth in the $y$%
-direction. However what is displayed in reality is what the A/D converter
will see, i.e.~the scale in slant range is measured as range delay. The
difference between distance and range delay is a factor of two, because the
pulse has to travel twice the distance. So the simulation window actually
uses range delay $=\frac{c}{2d}$ (in seconds) as the $x$-axis, but shows a
scale which corresponds to the actual distance to the target. An example: a
pulse of length $1\,u$s is $300$\thinspace m long. If the A/D frequency is $%
500$\thinspace MHz, the returning pulse will cover over $500$ samples. One
sample corresponds to $0.3$\thinspace m $\left( \frac{c}{2\cdot f_{ad}}%
\right) $ in distance and therefore the pulse will have a length of $%
500\cdot 0.3\,$m $=150\,$m on the display (and not the actual $300$%
\thinspace m). This might sound confusing, but the factor of $2$ simply
originates from the fact that the beam has to travel twice the actual
distance (to the target and back). If the return is converted into range by
multiplying by the speed of light, the target will appear at twice the
distance where it actually is, and therefore the range (and the pulse
length) is divided by a factor of $2$ to overcome this problem.

\subsection{Saving Data}

\label{savedata}After choosing the required window in range and azimuth, the
data can be saved in either text or binary format. The dialog shown in
Figure~\ref{savedata} will come up.

\FRAME{fhFU}{3.3529in}{4.0153in}{0pt}{\Qcb{``Save Data'' dialog}}{\Qlb{%
savedata}}{savedata.gif}{\special{language "Scientific Word";type
"GRAPHIC";maintain-aspect-ratio TRUE;display "ICON";valid_file "F";width
3.3529in;height 4.0153in;depth 0pt;original-width 550.3125pt;original-height
660.1875pt;cropleft "0";croptop "1";cropright "1";cropbottom "0";filename
'J:/SARSIM2/DOC/TEX/GRAPHICS/Savedata.gif';file-properties "XNPEU";}}

The dialog is divided into three sections:

\subsubsection{Simulation Window}

The simulation window determines the range in time (azimuth) and distance
(slant range) which will be saved to disk. The actual values were set in the
previous window (simulation window) and cannot be changed in this dialog.
The slant range is given in meters and represents the distance from the
radar. It can be converted into time (range delay) by using the simple
relationship $t=\frac{2d}{c}$. The azimuth range is measured in seconds and
corresponds to the time span for which the data will be recorded. If the PRF
is constant, the number of pulses can be determined by using the following
equation: 
\begin{equation}
\text{number of pulses }=(\text{EndTime}-\text{StartTime})\cdot \text{PRF}+1
\end{equation}

In non-SAR applications the term azimuth range is equivalent to the time
window (slow time) for which the radar is capturing data.

The sampling frequency determines the sampling rate of the received signal.
By default the sample frequency is set to the Nyquist frequency which is
calculated from either the pulse length or the chirp bandwidth for
monochrome and chirp pulses respectively. The sampling frequency can be
altered by changing the multiplication factor. Undersampling, i.e.~inserting
values below unity, is accepted.

Below that, the size of the file is estimated. In this case (see Figure~\ref
{savedata}), there are $512$ sample points in slant range $(1279$\thinspace
m in range $\triangleq $ $\frac{2\cdot 1279}{c}=8.5352\,us$, $%
8.5352\,us\cdot 120\,$MHz $=1024)$ and $512$ pulses in azimuth ($1.279$
seconds at $400$\thinspace Hz PRF (inclusive)). For ASCII files, the program
can only estimate the actual space needed for each I/Q pair, as trailing
zeroes are not saved. However if the file is saved in binary format, the
size given will be precise.

\subsubsection{A/D Converter}

The next box requires two inputs, namely the A/D sampling accuracy in bits
and the value of the least significant bit (LSB) in millivolt. If the value
is not divisible by $8$, the remaining bits will be zero-padded. The LSB
value is estimated from the maximum magnitude occurring in the simulation
window. It will not be absolutely accurate since the sampling frequency of
the screen window might be less than the sampling frequency used for saving
the file. It is possible to change the LSB value by selecting the ``Set by
user'' check box. After the file has been saved, the program will show how
much dynamic range has actually been used.

\subsubsection{File}

The last box contains the file name and the format in which the data should
be saved in. If the file is saved in binary format, each I and Q value is
represented by an integer number of bytes with zero-padding applied if
necessary. For example $12$ bit A/D accuracy will require $2$ bytes for each
I and Q value, with the four least-significant bits set to $0$. The offset
is $\left( 2^{x-1}-1\right) $ with $x$ being the number of A/D bits (for
example $8$ bits correspond to an offset of $127$, i.e. $0$ Volts = $127$).
If the file is saved in text (ASCII) format the number (as calculated
before) will be written as a proper number.

An example is given now:

In the example above the maximum value in the simulation window was $%
6.39369\cdot 10^{-5}$\thinspace mV. Using $8$ bits per value, this means the
least significant value will be $\frac{6.39369\cdot 10^{-5}}{2^{8-1}-1}=%
\frac{6.39369\cdot 10^{-5}}{127}=5.0344\cdot 10^{-7}$\thinspace mV as is
also shown in Figure~\ref{savedata}. If for example the I-value at some
point is $1.2\cdot 10^{-5}$\thinspace mV, then the value written to disk
will be $(2^{8-1}-1)+\func{round}\left( \frac{1.2\cdot 10^{-5}}{5.0344\cdot
10^{-7}}\right) =127+37=164$, where the first part of the formula represents
the offset value. In binary mode the character corresponding to $164$ will
be written to file, while in ASCII mode the three separate digits ``1'',
``6'' and ``4'' will be saved.

There is the option to save the current simulation window (i.e.~the window
range, sampling frequency, etc.) in the script file so that it can be
recalled quickly at a later stage. This is done by setting the ``Save to
script only'' or the ``Save to both'' check box. If it is set to ``Save to
script only'' it means the file will be not written to disk, however the
simulation parameters will be added to the script file. This will be
explained in more detail in Chapter~\ref{chapter:script}. If it is set to
``Save simulation output to disk only'', no simulation entry will be added
to the script file. ``Save to both'' adds the simulation to the script file
and also saves the actual data to disk.

After selecting the ``Save'' option, the file will be saved, showing a
progress bar. The saving process can always be interrupted by pressing the
cancel button.

\subsection{Data file structure}

The data is saved in a complex format where each sample point is represented
by an inphase component (I) and a quadrature component (Q). The structure is
shown in Figure~\ref{filestruc}.

\FRAME{fhFU}{7.8507cm}{5.2258cm}{0pt}{\Qcb{Data file structure}}{\Qlb{%
filestruc}}{data save structure.eps}{\special{language "Scientific
Word";type "GRAPHIC";maintain-aspect-ratio TRUE;display "ICON";valid_file
"F";width 7.8507cm;height 5.2258cm;depth 0pt;original-width
277.0625pt;original-height 183.6875pt;cropleft "0";croptop "1";cropright
"1";cropbottom "0";filename 'J:/SARSIM2/DOC/TEX/GRAPHICS/DATA SAVE
STRUCTURE.EPS';file-properties "XNPEU";}}

The I and Q values are written in turns. Each line in the file corresponds
to the return of one pulse, with the ``earliest'' pulse written first.
Within each line the sampled values are written as IQ pairs, the ``earlier''
(closer to the radar) samples written first. These pairs are separated by
commas if the file is saved as text, but have no separation character if the
file is saved in binary format. In text mode after each pulse a ``new line''
character is inserted. For the parameters shown in Figure~\ref{savedata}
there would be $1024$ IQ pairs ($2048$ numbers) for each line and a total of 
$512$ lines.

\section{User-defined Functions}

\label{userfunc}The radar simulator is highly configurable. There are $12$
functions which can be defined completely arbitrarily, for example the pulse
shape, the centre frequency of every pulse, the PRI (pulse repetition
interval) between any $2$ pulses, the trajectory of moving platforms, the
antenna gain patterns, matched filter windows and some more. These functions
can either be defined by entering the sample points straight into the
simulator, or by specifying an external data file.

\subsection{ The data entry window}

A standardised input screen is used for all of them, which is shown in
Figure~\ref{userdef}.

\FRAME{fhFU}{4.19in}{3.3278in}{0pt}{\Qcb{A user-defined function}}{\Qlb{%
userdef}}{userdef.gif}{\special{language "Scientific Word";type
"GRAPHIC";maintain-aspect-ratio TRUE;display "ICON";valid_file "F";width
4.19in;height 3.3278in;depth 0pt;original-width 0pt;original-height
0pt;cropleft "0";croptop "1";cropright "1";cropbottom "0";filename
'D:/SARSIM2/DOC/TEX/GRAPHICS/Userdef.gif';file-properties "XNPEU";}}

In this case the window for defining a trajectory of a platform is shown.
There are three variables to be defined for the trajectory, i.e.~the $x$, $y$
and $z$ coordinates vs time. To define a function, at least one point
(coordinate vs time) has to be specified for each of the three variables.
The program will interpolate between the given points. If only one point is
specified (for example $500$\thinspace m at $t=0$ seconds), the coordinate
will stay constant ($500$\thinspace m) and for all given times the graph
will be a (horizontal) straight line passing through $(0,500)$. If two
points are given, the graph will also be a straight line passing through
both points, but will not necessarily be horizontal. For more than $2$
points, the program will use the selected interpolation method. Note that
the points do not have to be defined with constant intervals for the $x$%
-coordinate, for example $(1,3)$, $(2,5)$ and $(20,10)$ is perfectly
acceptable.

There are currently $3$ interpolation methods implemented:

\begin{enumerate}
\item  \textbf{Cubic splines: }This will use cubic splines to interpolate
between points. This interpolation is computationally intensive and should
be used only for less than $1000$ points or so.

\item  \textbf{Low pass filter: }In this case the defined points can be
considered as impulses to a low-pass filter with a highest frequency
response given by the inverse of the average distance between $2$ successive
points. Note that the function might not touch all given points if their $x$%
-interval is not constant.

\item  \textbf{Linear: }A straight line interpolation between $2$ successive
points is implemented, as shown in Figure~\ref{userdef}.
\end{enumerate}

A comparison between the different methods is given in Figure~\ref
{interpolation}.

\FRAME{fhFU}{5.7968in}{2.4898in}{0pt}{\Qcb{The different interpolation
methods}}{\Qlb{interpolation}}{interpolation.emf}{\special{language
"Scientific Word";type "GRAPHIC";maintain-aspect-ratio TRUE;display
"ICON";valid_file "F";width 5.7968in;height 2.4898in;depth
0pt;original-width 417.1875pt;original-height 178.1875pt;cropleft
"0";croptop "1";cropright "1";cropbottom "0";filename
'D:/SARSIM2/DOC/TEX/GRAPHICS/INTERPOLATION.EMF';file-properties "XNPEU";}}

For the example in Figure~\ref{userdef}, $3$ samples are given. To create a
platform moving with constant velocity it would be sufficient to specify two
samples.

A detailed description of every part of the window is given below:

\begin{itemize}
\item  \textbf{Source: }The data can be either stored in a separate file or
inline (within the script file). For large data sets it might be more
practical to use external data files. However for simple simulations the
``inline'' option is more compact. The structure is described in Chapter~\ref
{chapter:script}.

\item  \textbf{Display:} Depending on the data type, there are up to three
variables which can be defined simultaneously. For the above example three
coordinates need to be defined. If there is more than one variable, one can
switch between them by clicking on the specific radio button.

\item  \textbf{Edit:} The data can be entered or modified in this box. The 
\textbf{Number of samples} edit box defines the total number of samples
specified for the given variable (they might be different for another
variable, for example you might want to specify $5$ points for the $x$%
-coordinate, but only $1$ for the $y$ and $z$ coordinate). Below that a list
of $x$ and $y$ coordinates is given for the number of specified points.
After entering the values press \textbf{Update} to update the graph on the
right. When entering more than $20$ values or so it is more practical to
read external text files. Points can be deleted or added at the current
cursor position.

\item  \textbf{Window: }Here the current range of the graph shown can be
set. Some data types are limited in range.

\item  It is possible to \textbf{zoom in} by selecting an area on the graph
with the left mouse button. This is done by pushing and holding the left
mouse button on the desired corner and releasing it on the opposite corner.

\item  \textbf{Zooming out} by a factor of two in either direction is
achieved by right-clicking anywhere on the graph.

\item  Points can be \textbf{dragged} to a new location on the graph by
pushing and holding the left mouse button on the desired point and moving it
to a new position.
\end{itemize}

All of the data types for which the ordinate increases in fixed steps of $1$%
, the ordinate will be ignored. The graph will not be a continuous line but
a bar graph. One example is the definition of the centre frequency for each
pulse. In this case the ordinate just represents the pulse number and will
be identical to the current sample number.

\subsection{External data files}

If the data is specified to be an external file, the structure of the data
depends on the functions to be defined, but is of the following general
format:

%TCIMACRO{
%\TeXButton{\tiny}{\tiny
%}}
%BeginExpansion
\tiny
%
%EndExpansion
\begin{verbatim}
[Number of samples for function 1], [X1], [Y1], [X2], [Y2], [X3], [Y3] ....[Xn],[Yn]
[Number of samples for function 2], [X1], [Y1], [X2], [Y2], [X3], [Y3] ....[Xn],[Yn]
[Number of samples for function 3], [X1], [Y1], [X2], [Y2], [X3], [Y3] ....[Xn],[Yn]
\end{verbatim}

%TCIMACRO{\TeXButton{\normalsize}{\normalsize}}
%BeginExpansion
\normalsize%
%EndExpansion

First the number of samples is given, then the $x$ and $y$ coordinates for
all the samples are given. This is repeated up to $3$ times depending on the
specific data type (for example position requires $3$ coordinates, others
only require $1$ or $2$). Comments can be inserted by using an exclamation
mark after which all text until the next line break is ignored. Commas or
spaces can be used as separators. As an example the data file for the points
given in Figure~\ref{interpolation} is given:

%TCIMACRO{
%\TeXButton{\tiny}{\tiny
%}}
%BeginExpansion
\tiny
%
%EndExpansion
\begin{verbatim}
! Data file for : Position X (m) vs Time (s), Position Y (m) vs Time (s), Position Z (m) vs Time (s)
! Structure : [No. Samples] [Time (s) 1] [Position X (m) 1] [Time (s) 2] [Position X (m) 2] ...
! [No. Samples] [Time (s) 1] [Position Y (m) 1] [Time (s) 2] [Position Y (m) 2] ...
! [No. Samples] [Time (s) 1] [Position Z (m) 1] [Time (s) 2] [Position Z (m) 2] ...
5
0 0, 1 1.5, 2 0, 3 -5, 4 -3.5
1
0 0
1
0 0
\end{verbatim}

%TCIMACRO{\TeXButton{\normalsize}{\normalsize}}
%BeginExpansion
\normalsize%
%EndExpansion

$5$ points have been define for the $x$ coordinate, and $1$ for the $y$ and $%
z$ coordinate. Data files can also be edited as described above and the
changes can be written to disk by clicking the \textbf{Save} button.

\section{Viewing Simulation Files}

A powerful image viewer has been integrated into Sarsim. It can be found
under \textbf{Display} / \textbf{Simulation Output}. The features of this
viewer include:

\begin{itemize}
\item  No limit in the file size of the image to be viewed. The file is
loaded ``on-the-fly'' and only the displayed portion is kept in memory.

\item  Import of Sun-Raster Files, Sarsim simulation files and user-defined
files.

\item  Thumbnail overview image shown.

\item  Slices in range or azimuth.

\item  Several colour palettes available with slider controls for
brightness, contrast and saturation.

\item  Automatic display of I,Q, phase and magnitude for selected sample.

\item  Usual zooming and panning controls.
\end{itemize}

The dialog displayed in Figure~\ref{loadsim} appears after selecting the
menu option \textbf{File} / \textbf{Open}.

\FRAME{fhFU}{2.6662in}{2.4405in}{0pt}{\Qcb{Load file dialog}}{\Qlb{loadsim}}{%
loadsim.gif}{\special{language "Scientific Word";type
"GRAPHIC";maintain-aspect-ratio TRUE;display "ICON";valid_file "F";width
2.6662in;height 2.4405in;depth 0pt;original-width 0pt;original-height
0pt;cropleft "0";croptop "1";cropright "1";cropbottom "0";filename
'D:/SARSIM2/DOC/TEX/GRAPHICS/Loadsim.gif';file-properties "XNPEU";}}

\begin{itemize}
\item  \textbf{Filename}: The filename of the image file. Use the \textbf{%
Find} button to browse for a file.

\item  \textbf{Format}: Several formats are supported:

\begin{enumerate}
\item  Sun-Raster-File format: These files have a header defining their size
and other parameters.

\item  Sarsim simulation: If there is a \textbf{\$SIMULATION} entry in the
script file with the given image filename as output name, it will use the
specified parameters.

\item  Custom: A number of parameters have to be specified for custom files:

\begin{itemize}
\item  Header: Header size in bytes, data in header will be ignored.

\item  Width: Width of image in pixel (corresponds to slant range)

\item  Height: Height of image in pixel (corresponds to azimuth range)

\item  Bytes per Value: Number of bytes for each pixel. Note: If samples are
complex this will be the number of bytes for both I and Q values together.

\item  Offset: The offset which will be subtracted from each sample (to get
negative numbers). For example single byte values have a range of 0 to 255.
An offset of 127 will map this to $-127$ to $+128$.

\item  Data Type: Either magnitude (single values) or complex. Complex
numbers need to have the I value first, i.e.~IQIQ.

\item  Byte Order: Specify if least-significant byte comes first or not.
\end{itemize}
\end{enumerate}
\end{itemize}

After successfully loading the file, the window will look similar to the one
shown in Figure~\ref{viewer}.

\FRAME{fhFU}{3.6409in}{4.0153in}{0pt}{\Qcb{The image viewer window (showing
slices in azimuth)}}{\Qlb{viewer}}{viewer.gif}{\special{language "Scientific
Word";type "GRAPHIC";maintain-aspect-ratio TRUE;display "ICON";valid_file
"F";width 3.6409in;height 4.0153in;depth 0pt;original-width
0pt;original-height 0pt;cropleft "0";croptop "1";cropright "1";cropbottom
"0";filename 'D:/SARSIM2/DOC/TEX/GRAPHICS/Viewer.gif';file-properties
"XNPEU";}}On the left, bottom side the complete image is shown. A white
rectangle marks the zoomed section, shown on the right. The data can be
displayed as:

\begin{itemize}
\item  Colour pixels: Each value corresponds to a certain colour (see colour
palette)

\item  Slices in $x$: This will show the data as graphs, ranging from top to
bottom.

\item  Slices in $y$: This will show the data as graphs, ranging from left
to right.
\end{itemize}

\section{Looking at Script Files}

Script files are used to describe the complete geometry of the simulation
and they also store the parameters of files written to disk. Sometimes it is
useful to quickly take a look at this script file. It can be displayed by
selecting \textbf{Display} / \textbf{Script File}. The file displayed will
be identical to the file which will be stored on disk after selecting 
\textbf{File} / \textbf{Save} on the main window. An example is shown in
Figure~\ref{scriptv}.

\FRAME{fhFU}{4.209in}{3.096in}{0pt}{\Qcb{The script file viewer}}{\Qlb{%
scriptv}}{scriptv.gif}{\special{language "Scientific Word";type
"GRAPHIC";maintain-aspect-ratio TRUE;display "ICON";valid_file "F";width
4.209in;height 3.096in;depth 0pt;original-width 1008pt;original-height
740pt;cropleft "0";croptop "1";cropright "1";cropbottom "0";filename
'J:/SARSIM2/DOC/TEX/GRAPHICS/Scriptv.gif';file-properties "XNPEU";}}

\section{Changing Focus}

\label{focuspoint}It is possible to change the focus point of the main
screen from the origin of the ``Earth'' coordinate system to somewhere else.
Either a point other than the origin can be selected or the focus can follow
a different platform. This is useful if one needs to see how an object moves
seen relative to some platform.

\section{Investigating the Geometry for every Pulse}

By selecting \textbf{Geometry} on the main menu it is possible to have a
look at the internal variables used for each simulation. This data can also
be saved in text format, so that it can be imported into other data analysis
programs. It is also a very useful ``debugging'' tool.

The purpose of this command is to show the relative locations and distances
between the radar and all given point targets for each pulse for the given
simulation period. An example will clarify this. Suppose you want to model
your own (very complicated) pulse shape and want to use Sarsim only to do
the geometry calculations, i.e.~distance and radial velocity for each point
target seen by the radar, this is the tool to use. The output format will be
a table of the general form shown in Table~\ref{tab:two}

%TCIMACRO{\TeXButton{B}{\begin{table}[tbp] \centering}}
%BeginExpansion
\begin{table}[tbp] \centering%
%EndExpansion
\begin{tabular}{|c|c|c|c|c|c|c|}
\hline
{\tiny Pulse No.} & {\tiny Pulse transmittion} & {\tiny PT1 Distance} & 
{\tiny PT1 Radial} & {\tiny PT2 Distance} & {\tiny PT2 Radial} & {\tiny PT3
etc} \\ 
& {\tiny time (s)} & {\tiny (m)} & {\tiny Velocity (m/s)} & {\tiny (m)} & 
{\tiny Velocity (m/s)} &  \\ \hline
{\tiny 0} & {\tiny 1.200} & {\tiny 32.323} & {\tiny 2.312} & {\tiny 34.411}
& {\tiny 1.212} & {\tiny etc.} \\ \hline
{\tiny 1} & {\tiny 1.202} & {\tiny 32.343} & {\tiny 2.323} & {\tiny 34.422}
& {\tiny 1.222} & {\tiny etc.} \\ \hline
{\tiny 2} & {\tiny 1.202} & {\tiny 32.363} & {\tiny 2.334} & {\tiny 34.433}
& {\tiny 1.232} & {\tiny etc.} \\ \hline
\end{tabular}
\caption{Distance and radial velocity for each point target\label{tab:two}}%
%TCIMACRO{\TeXButton{E}{\end{table}}}
%BeginExpansion
\end{table}%
%EndExpansion

Note that in Table~\ref{tab:two} only a small subset of the variables have
been shown. A detailed list of all variables is given below.

\begin{itemize}
\item  Pulse No.: This gives each transmitted pulse a unique number. Pulse $0
$ (zero) will always be sent at simulation time $t=0$, the next pulse will
be $1$ etc. If a simulation extends over negative time, the pulse numbers
will be negative.

\item  Pulse transmitting time (seconds): Displays the exact time when the
pulse is transmitted (for example pulse $0$ will be transmitted at time $t=0$%
, pulse $1$ at time $t=$ PRI, $2$ at $t=2\cdot $ PRI, and so on, assuming
constant PRIs).

\item  Pulse frequency (GHz): Displays the exact frequency for each pulse
given in Gigahertz.

\item  Platform position $x,y,z$ (m): Displays the exact position in meters
for each defined platform at the instance of time where the pulse was sent.
The coordinates are relative to the world coordinate system. Note that a
radar will always reside on the origin of the platform, so the given
position will be identical to the position of a radar (if any) situated on
this specific platform.

\item  Platform velocity $x,y,z$ (m/s): Displays the exact velocity in
meters per second for each defined platform at the instance of time where
the pulse was sent. The velocity vector is relative to the world coordinate
system.

\item  Platform rotation $x,y,z$ (degrees): Displays the exact rotation of
the platform around the $3$ axes at the instance of time when the pulse was
sent.
\end{itemize}

\section{Help}

A slightly modified version of this entire dissertation is available under 
\textbf{Help} / \textbf{Contents}.
