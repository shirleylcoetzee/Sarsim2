%TCIDATA{LaTeXparent=0,0,thesis.tex}
                      

\chapter{Radar Theory applied to Simulator\label{chap:theory}}

This chapter gives a very brief overview of radar theory and how it is
implemented in the simulator.

Radars are electromagnetic devices used for detection of targets by
radiating electromagnetic energy and examining the reflected energy. A
target can be described as an object which interferes with the transmitted
wave and reflects part of its energy. For this simulator, targets occupy an
infinitesimally small space, therefore they are called point targets
(PTs).Most radars work in pulsed mode, i.e. they send out short burst of
energy, with relatively long delays in-between. The period between these
pulses is called the \textit{pulse repetition interval} (PRI), but usually
the reciprocal, namely the \textit{pulse repetition frequency }(PRF), is
specified.

\FRAME{fhFU}{3.4134in}{1.5108in}{0pt}{\Qcb{A simple radar setup}}{\Qlb{%
antennapt}}{antennapt.eps}{\special{language "Scientific Word";type
"GRAPHIC";maintain-aspect-ratio TRUE;display "ICON";valid_file "F";width
3.4134in;height 1.5108in;depth 0pt;original-width 361.375pt;original-height
157.625pt;cropleft "0";croptop "1";cropright "1";cropbottom "0";filename
'D:/SARSIM2/DOC/TEX/GRAPHICS/ANTENNAPT.EPS';file-properties "XNPEU";}}

Consider the setup shown in Figure~\ref{antennapt} with a single point
target. A pulse is sent out at time $t_{0}$ and the return is received after 
$\Delta t$ seconds: 
\begin{equation}
\Delta t=\frac{2\cdot d}{c}  \label{equ:delta_t}
\end{equation}
where $c$ is the speed of light and $d$ is the distance to the target. For
this simulator the received signal will be a replica of the transmitted
signal, although the power will be a fraction of the original pulse. The
received power $P_{rx}$ of the reflected signal at the radar will be 
\begin{equation}
P_{rx}=\frac{P_{tx}\,G_{t}\,G_{r}\,\lambda ^{2}\,\sigma }{\left( 4\pi
\right) ^{3}\,d^{4}}
\end{equation}
where $P_{tx}$ is the transmitted power, $G_{t}$ and $G_{r}$ the transmit
and receive antenna gain, $\lambda $ the wavelength, $\sigma $ the cross
section area of the point target and $d$ the distance between radar and
target.

Normally many pulses are transmitted as shown in Figure~\ref{radarpulses}.

\FRAME{fhFU}{5.4552in}{1.3993in}{0pt}{\Qcb{Pulsed operation}}{\Qlb{%
radarpulses}}{radarpulse.eps}{\special{language "Scientific Word";type
"GRAPHIC";maintain-aspect-ratio TRUE;display "ICON";valid_file "F";width
5.4552in;height 1.3993in;depth 0pt;original-width 361.375pt;original-height
90.0625pt;cropleft "0";croptop "1";cropright "1";cropbottom "0";filename
'D:/SARSIM2/DOC/TEX/GRAPHICS/RADARPULSE.EPS';file-properties "XNPEU";}}The
point target will return an echo after some time $\Delta t$ which depends on
the distance of the target from the radar according to Equation~\ref
{equ:delta_t}. Usually one is not interested in the complete range between
two pulses, but only a small fraction of it, which is set by a range gate.
Obviously the range delay and returned power can be calculated by hand for
simple stationary targets, but it will get very tedious for moving targets,
especially with the addition of noise and motion jitter.

This simulator is capable of generating an array of complex values which
correspond to the output of the radar at a certain timespan. For this
simulator the nomenclature of SAR literature has been used, so the interval
for which data will be sampled (range gate) is called slant range. The time
range for which pulses are sent out and the return is sampled is called
azimuth range. An example of a single point target which moves with constant
velocity past a radar is shown in Figure~\ref{windowex}.

\FRAME{fhFU}{4.0214in}{3.141in}{0pt}{\Qcb{An example of a simulation output
of a moving point target}}{\Qlb{windowex}}{windowex.gif}{\special{language
"Scientific Word";type "GRAPHIC";maintain-aspect-ratio TRUE;display
"ICON";valid_file "F";width 4.0214in;height 3.141in;depth 0pt;original-width
0pt;original-height 0pt;cropleft "0";croptop "1";cropright "1";cropbottom
"0";filename 'D:/SARSIM2/DOC/TEX/GRAPHICS/Windowex.gif';file-properties
"XNPEU";}}A target moves past the radar, being closest at time $t=0$. The $x$%
-axis represents distance from the radar, and the $y$-axis represents time.
In this case chirp modulation has been used with a very low (unrealistic)
chirp bandwidth. The simulation shows the raw return (no range compression
applied). Note that the signal magnitude decreases as the distance to the
radar increases.

This radar simulator therefore creates 2-dimensional arrays of complex
numbers corresponding to the output of range versus time for a specific
radar. It calculates the relative distances between the radar(s) and point
targets for the specified time period. Depending on these distances, scaled
copies of the defined pulse are inserted into the output array. The raw
return can be range compressed, i.e.~every pulse is convolved with its
replica.
