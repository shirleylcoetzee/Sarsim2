%TCIDATA{LaTeXparent=0,0,THESIS.TEX}
                      

\chapter{Sarsim Internals\label{chapter:internals}}

In this chapter the formulas behind Sarsim will be explained in detail. The
functions which depend on each other will be presented in order of
dependence.

\section{General Function and Variable Definitions}

\vspace {0.3cm}

$PNo=$ pulse number (integer $\geqq $ 0)

\vspace{0.3cm}

$TNo=$ target number (integer $\geqq $ 0)

\vspace {0.3cm}

$PFNo=$ platform number (integer $\geqq $ 0)

\vspace{0.3cm}

$floor(x)=$ integer part of number

\vspace {0.3cm}

$\func{fmod}(x,y)=$ the remainder $f$, where $x=ay+f$ for some integer $a$,
and $0<f<y$ (modulus)

\vspace {0.3cm}

\[
\text{{\tiny RotMatrix(a1,a2,a3) = }}\left[ 
\begin{array}{lll}
\text{{\tiny cos(a2)cos(a3)}} & \text{{\tiny -cos(a2)sin(a3)}} & \text{%
{\tiny sin(a2)}} \\ 
\text{{\tiny cos(a1)sin(a3)+sin(a1)sin(a2)cos(a3)}} & \text{{\tiny %
cos(a1)sin(a3)+sin(a1)sin(a2)cos(a3)}} & \text{{\tiny -sin(a1)cos(a2)}} \\ 
\text{{\tiny sin(a1)sin(a3)-cos(a1)sin(a2)cos(a3)}} & \text{{\tiny %
sin(a1)cos(a3)+cos(a1)sin(a2)sin(a3)}} & \text{{\tiny cos(a1)cos(a2)}}
\end{array}
\right] 
\]

\section{Pulse Calculations}

Let us assume we want to calculate the return for the following window in
azimuth and slant range as shown in Figure~\ref{sim-window}.

\FRAME{fhFU}{1.9242in}{1.6025in}{0pt}{\Qcb{Simulation window}}{\Qlb{%
sim-window}}{int-window.eps}{\special{language "Scientific Word";type
"GRAPHIC";maintain-aspect-ratio TRUE;display "ICON";valid_file "F";width
1.9242in;height 1.6025in;depth 0pt;original-width 0pt;original-height
0pt;cropleft "0";croptop "1";cropright "1";cropbottom "0";filename
'D:/SARSIM2/DOC/TEX/GRAPHICS/INT-WINDOW.EPS';file-properties "XNPEU";}}

The variables \textit{AzimuthStart} and \textit{AzimuthEnd} are given in
seconds, while \textit{SlantStart} and \textit{SlantEnd} are given in meters.

\subsection{FindPulseSendTime Function}

This function is defined in \texttt{engine.cpp}. It returns the time (in
seconds) at which the given pulse is sent (i.e.~the beginning of the pulse).
Pulse $0$ is always sent at time $t=0$. For a constant PRF the function is : 
\begin{equation}
PulseSendTime=\frac{PulseNo}{\text{PRF}}
\end{equation}

For user-defined PRIs, the function is more complicated. The slightly
modified (for clarity reasons) source code is given here:

%TCIMACRO{
%\TeXButton{\tiny}{\tiny
%}}
%BeginExpansion
\tiny
%
%EndExpansion
\begin{verbatim}
long i,FracPRINo;
double frac, ipart, FracPRI;
double SumPRI=0; // sum of all defined PRI's
// find sum of all PRI's defined (i.e. PRI0+PRI1+PRI2 etc.)
for (i=0;i<n;i++)
SumPRI += PRIArray[i];
// find out how many complete PRI cycles there are in the given
// PulseNo (1 cycle = sum of all defined PRI's)
frac = modf(double(PulseNo)/double(n), &ipart);
// calculate the number of remaining PRI's
FracPRINo = long(round(frac * double(n)));
// if PulseNo <0 goto the next lower integral time (ipart -= 1)
// and add positive PRI's from there
if (FracPRINo < 0)
FracPRINo += n;
ipart -= 1;
FracPRI = 0;
for (i=0;i<FracPRINo;i++)
FracPRI += PRIArray[i];
return (ipart * SumPRI + FracPRI);
\end{verbatim}

%TCIMACRO{\TeXButton{\normalsize}{\normalsize}}
%BeginExpansion
\normalsize%
%EndExpansion

\subsection{FindPulsesInRange Function}

First the exact time when each pulse is sent out needs to be calculated.
This is important for calculating the relative distances between the radar
and the point targets. Pulses are numbered such that pulse $0$ will be sent
at simulation time $t=0$, pulse $1$ will be sent at time $t=$ 0 $+$ PRI$[0]$%
, pulse $2$ will be sent at time $t=$ 0 $+$ PRI$[0]\,$+ PRI$[1]$ etc.
Negative times (and pulse numbers) are possible, for example pulse $-1$ will
be sent at time $t=$ 0 $-$ PRI$\left[ n-1\right] $. In Sarsim it is possible
to define the interval between any two pulses. Let us assume this data is
given in the array called $PRIArray[x]$, where $x$ ranges from $0$ to $%
\left( n-1\right) $, where $n$ is the number of defined PRIs. The array
wraps around such that PRI$[x]=PRIArray\left[ \func{modulus}(x,n)\right] $.
Note that all times are given in seconds. Figure~\ref{pulses} will clarify
what was explained above.

\FRAME{fbFU}{4.8023in}{1.2912in}{0pt}{\Qcb{The ``PulseSendTime''}}{\Qlb{%
pulses}}{int-pulses.eps}{\special{language "Scientific Word";type
"GRAPHIC";maintain-aspect-ratio TRUE;display "ICON";valid_file "F";width
4.8023in;height 1.2912in;depth 0pt;original-width 0pt;original-height
0pt;cropleft "0";croptop "1";cropright "1";cropbottom "0";filename
'D:/SARSIM2/DOC/TEX/GRAPHICS/INT-PULSES.EPS';file-properties "XNPEU";}}

%TCIMACRO{\TeXButton{pagebreak}{\pagebreak}}
%BeginExpansion
\pagebreak%
%EndExpansion

There are two cases:

\begin{enumerate}
\item  PRF is constant: 
\begin{eqnarray}
FirstPulse &=&\func{ceil}\left( AzimuthStart\cdot \text{PRF}\right)  \\
LastPulse &=&\func{floor}\left( AzimuthEnd\cdot \text{PRF}\right) 
\end{eqnarray}

\item  User-defined PRIs: The function is more complicated. The slightly
modified (for clarity reasons) source code is given here:
\end{enumerate}

%TCIMACRO{
%\TeXButton{\tiny}{\tiny
%}}
%BeginExpansion
\tiny
%
%EndExpansion
\begin{verbatim}
long i;
// user defined PRI
double SumPRI=0; // sum of all defined PRI's
// find sum of all PRI's defined (i.e. PRI0+PRI1+PRI2 etc.)
for (i=0;i<n;i++)
SumPRI += PRIArray[i];
// estimate what number the first pulse will have
FirstPulse = (floor((TimeStart / SumPRI)+ROUNDERROR)*n)-1;
// and now find the exact one
while (FindPulseSendTime(FirstPulse) < TimeStart)
{
FirstPulse++;
}
LastPulse = (floor((TimeEnd / SumPRI)+ROUNDERROR)*n)-1;
while (FindPulseSendTime(LastPulse) < TimeEnd)
}
LastPulse++;
}
// overshot by one, so subtract one again
if (*LastPulse > 0) (*LastPulse)--;
\end{verbatim}

%TCIMACRO{\TeXButton{\normalsize}{\normalsize}}
%BeginExpansion
\normalsize%
%EndExpansion

\subsection{FindPlatformPosition}

This function finds the platform position at a give time $t$.

\subsection{FindPlatformVelocity}

This function finds the platform velocity at a give time $t$.

\subsection{FindPlatformRotation}

This function finds the platform rotation at a give time $t$.

\subsection{The frequency of each pulse}

\textbf{Single frequency case:}\emph{\ } 
\begin{equation}
PulseFreq[PNo]=\text{specified in RADAR dialog 1}
\end{equation}

\textbf{Stepped frequency case:}\emph{\ } 
\begin{eqnarray}
PulseFreq[PNo] &=&StartFreq  \nonumber \\
&&+~floor\,(fmod\left( \frac{PNo}{FreqSteps\cdot PulsesPerFreq},1\right) 
\nonumber \\
&&\cdot ~FreqSteps+ROUNDERROR)\cdot StepSize
\end{eqnarray}

Note that $ROUNDERROR$ (a small number) needed to be included to overcome
rounding problems, for example sometimes $\frac{12}{6}$ would give
1.999999etc which is incorrectly rounded to $1$.

\textbf{User-defined case:}\emph{\ } 
\begin{equation}
PulseFreq[PNo]=DataArray[fmod[PNo],arraysize]
\end{equation}

\subsection{The time when each pulse is transmitted}

\textbf{Constant PRF:}\emph{\ } 
\begin{equation}
PulseSendTime[PNo]=\frac{PNo}{PRF}
\end{equation}

\textbf{User-defined PRIs:}\emph{\ } 
\begin{equation}
PulseSendTime[PNo]=\text{see source code}
\end{equation}

\subsection{Range delay}

\begin{equation}
RangeDelay[TNo][PNo]=\frac{2\cdot TargetDist}{\text{{\tiny LIGHT\_SPEED}}%
-TargetRadialVel[TNo][PNo]}
\end{equation}

\subsection{CalcGeometry Function}

This function calculates the signal amplitude and range delay for the return
of all pulses and point targets of interest contained in the simulation
window.

\begin{itemize}
\item  Find \textit{Firstpulse} and \textit{LastPulse} by using function%
%TCIMACRO{\TeXButton{newline}{\newline} }
%BeginExpansion
\newline%
%EndExpansion
$FindPulsesInRange(AzimuthStart,AzimuthEnd)$

\item  Calculate the number of pulses which need to be calculated: 
\begin{equation}
PulseNo=(LastPulse-FirstPulse)+1
\end{equation}

\item  Create array which contains the times for which each pulse is sent: 
\begin{equation}
PulseSendTime[PNo]=FindPulseSendTime(PNo+FirstPulse)
\end{equation}

\item  Create 2D arrays which contain the position, velocity and rotation
for all platforms for all pulses: 
\begin{eqnarray}
\text{{\tiny PlatformPos[PFNo][PNo]}} &=&\text{{\tiny %
FindPlatformPosition(PFNo,PulseSendTime[PNo])}}  \nonumber \\
\text{{\tiny PlatformVel[PFNo][PNo]}} &=&\text{{\tiny %
FindPlatformVelocity(PFNo,PulseSendTime[PNo])}} \\
\text{{\tiny PlatformRot[PFNo][PNo]}} &=&\text{{\tiny %
FindPlatformRotation(PFNo,PulseSendTime[PNo])}}  \nonumber
\end{eqnarray}

\item  Compute the return gain factor independent of time: 
\begin{equation}
GainFactor=\frac{\frac{c}{Radar->StartFreq}\cdot \sqrt{Radar->PowerOutput}}{%
(4\cdot \pi )^{1.5}\cdot \sqrt{Radar->Losses}}
\end{equation}
\end{itemize}

with Radar-\TEXTsymbol{>}StartFreq being the centre frequency given in Hz,
Radar-\TEXTsymbol{>}PowerOutput given in Watt and Radar-\TEXTsymbol{>}Losses
given as a unitless factor.

\begin{itemize}
\item  For a sinusoidal antenna the gain for a certain offset angle can be
calculated using function $SinAntennaGain$.

\item  The combined antenna gain is given by $\sqrt{AntennaGainT\cdot
AntennaGainR}$ where $AntennaGainT$ is transmitter antenna gain and $%
AntennaGainR$ is the receiver antenna gain which can be calculated from: 
\begin{equation}
\text{{\tiny AntennaGainT =} }\left\{ 
\begin{array}{l}
\text{{\tiny 1~ for isotropic antennas}} \\ 
\text{{\tiny SinAntennaGain(OffsetAzi,AziBeamWidthT)}} \\ 
\cdot ~\text{{\tiny SinAntennaGain(OffsetElev,ElevBeamWidthT)~ for
sinusoidal antennas}}
\end{array}
\right. 
\end{equation}

\item  The return amplitude can be calculated with the formula 
\begin{equation}
ReturnAmp[TNo][PNo]=\frac{GainFactor\cdot AntennaGain\cdot \sqrt{RCS}}{%
TargetDist^{2}}
\end{equation}
\end{itemize}

\subsection{CalcOnePulse Function}

\subsubsection{Chirp Modulation}

The chirp rate (= DelaySlope, Hz/s) is calculated as follows:

\begin{equation}
DelaySlope=\frac{ChirpBandWidth}{PulseWidth}
\end{equation}

\subsubsection{Monochrome Modulation}

For monochrome pulses the DelaySlope would be zero.

\begin{equation}
DelaySlope=0
\end{equation}

The range delay is the time (in seconds) needed for the pulse travelling
forth and back to the point target and is given by:

\begin{equation}
RangeDelay=\frac{2\cdot d}{c}
\end{equation}

where $d$ is the distance to the target in meters and $c$ is the speed of
light (=~299792500 m/s).

The position of the pulse relative to the target is specified by

\begin{equation}
PulseCentre=\left\{ 
\begin{array}{l}
0\text{ ~{\tiny if the pulse is at the beginning of the point target}} \\ 
\frac{PulseWidth}{2}~\text{{\tiny \ if the pulse is at the centre of the
point target}}
\end{array}
\right.
\end{equation}

and explained in Figure~\ref{pulsepos}. For real radars you would receive
the pulse ``after'' the point target location, however for simulations it is
sometimes more convenient to have the point target in the centre. All it
really means is that the output array will be shifted by half a pulsewidth
in range.

\FRAME{fhFU}{5.5936in}{1.4546in}{0pt}{\Qcb{Point target at the start of the
pulse and at the centre of the pulse}}{\Qlb{pulsepos}}{pulsepos.eps}{%
\special{language "Scientific Word";type "GRAPHIC";maintain-aspect-ratio
TRUE;display "ICON";valid_file "F";width 5.5936in;height 1.4546in;depth
0pt;original-width 361.375pt;original-height 90.5pt;cropleft "0";croptop
"1";cropright "1";cropbottom "0";filename
'D:/SARSIM2/DOC/TEX/GRAPHICS/Pulsepos.eps';file-properties "XNPEU";}}

For the following calculations it is assumed that the point target position
corresponds to the beginning of the pulse. A certain time range needs to be
sampled denoted by $SlantStartTime$ and $SlantStartEnd$, both measured in
seconds. The sampling frequency is $f_{s}$. The pulse is situated from $%
RangeDelay$ to $RangeDelay+Pulsewidth$, both variables given in seconds. The
time axis is shifted by an amount of $\left( RangeDelay-\frac{1}{2}%
Pulsewidth\right) $ as shown in Figure~\ref{pulsedef}, such that the
variable $t$ goes from $-\frac{1}{2}Pulsewidth$ to $+\frac{1}{2}Pulsewidth$
over the pulse range.

\FRAME{fhFU}{5.8574in}{1.9545in}{0pt}{\Qcb{Positioning of pulse in range}}{%
\Qlb{pulsedef}}{pulsedef.eps}{\special{language "Scientific Word";type
"GRAPHIC";maintain-aspect-ratio TRUE;display "ICON";valid_file "F";width
5.8574in;height 1.9545in;depth 0pt;original-width 361.375pt;original-height
118.4375pt;cropleft "0";croptop "1";cropright "1";cropbottom "0";filename
'D:/SARSIM2/DOC/TEX/GRAPHICS/Pulsedef.eps';file-properties "XNPEU";}}

The frequency modulation (chirp rate for chirp pulses) can be calculated as:

\begin{equation}
Mod=DelaySlope\cdot \frac{1}{2}\cdot t^{2}
\end{equation}

For monochrome pulses this value would be zero (DelaySlope = $0$).

The instantaneous frequency of the returned pulse at some point $t$ ($t=0$
at beginning of pulse as shown in Figure~\ref{pulsedef}) is:

\begin{equation}
Freq(t)=\stackunder{\text{Modulation}}{\underbrace{DelaySlope\cdot t}}%
\stackunder{\text{Frequency shift due to Doppler}}{\underbrace{\,-\frac{%
2\cdot RadVel}{c}\cdot (Freq+DelaySlope\cdot t)}}
\end{equation}

The phase of the returned pulse is the integral with respect to time of the
frequency and can be calculated as follows:

\begin{equation}
Phase(t)=2\cdot \pi \cdot (\stackunder{\text{modulation}}{\underbrace{Mod}}%
\stackunder{\text{phase shift due to range}}{\underbrace{-\,(PulseFreq\cdot
RangeDelay)}}\stackunder{\text{phase shift due to Doppler}}{\underbrace{\,-%
\frac{2\cdot RadVel}{c}\cdot (Freq\cdot t+Mod)}}
\end{equation}

$Freq$ stands for the $PulseFreq[PNo][TNo]$ and specifies the centre
frequency of that specific pulse sent out, $RangeDelay$ is defined above and 
$RadVel$ specifies the radial velocity of the target.

From here the inphase and quadrature values are calculated simply by:

\begin{eqnarray}
I(t) &=&ReturnAmp\cdot \cos \left( Phase(t)\right) \\
Q(t) &=&ReturnAmp\cdot \sin \left( Phase(t)\right)
\end{eqnarray}
