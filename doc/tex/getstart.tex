%TCIDATA{LaTeXparent=0,0,THESIS.TEX}
                      

\chapter{Getting Started\label{chapter:start}}

This chapter introduces the radar simulator by setting up a very simple
simulation.

\section{The start-up window\label{chapter:quickstart}}

Start the program by executing ``sarsim2.exe'' either from the command
prompt or by double-clicking on the icon in Windows. A screen with a
coordinate-system on the right and a list of objects on the left will appear
as shown in Figure \ref{mainscre}.

\FRAME{fhFU}{5.8591in}{4.4486in}{0pt}{\Qcb{Main window}}{\Qlb{mainscre}}{%
mainscre.gif}{\special{language "Scientific Word";type
"GRAPHIC";maintain-aspect-ratio TRUE;display "ICON";valid_file "F";width
5.8591in;height 4.4486in;depth 0pt;original-width 0pt;original-height
0pt;cropleft "0";croptop "1";cropright "1";cropbottom "0";filename
'D:/SARSIM2/DOC/TEX/GRAPHICS/Mainscre.gif';file-properties "XNPEU";}}

This radar simulator uses three different kinds of objects: platforms, point
targets and radars.

\begin{itemize}
\item  \textbf{Platforms} are user-defined coordinate systems which can move
independently on any path seen relative to the ``Earth'' coordinate system
which will be explained below. All point targets or radars defined on that
platform will be stationary as seen from that coordinate system.

\item  \textbf{Point targets }are infinitely small points which reflect
electromagnetic energy. The amount of reflection depends on their surface
area and their directional vector. This will be explained in more detail in
Chapter~\ref{chapter:commands}.

\item  \textbf{Radars} are the actual energy-transmitting devices. They can
be positioned onto any platform (however their position is always at the
origin of that platform), and there are numerous parameters to be set which
will be explained in Chapter~\ref{chapter:commands}.
\end{itemize}

\subsection{The ``EARTH'' coordinate system}

The coordinate system shown on the start-up window is the most fundamental
one and all other objects will be positioned relative to it. It is
conveniently called ``Earth''. The origin of the ``Earth'' coordinate system
is at $x=0$, $y=0$ and $z=0$ with the red, green and blue lines representing
the $x$-axis, $y$-axis and $z$-axis respectively. The labelled ends are
pointing in the positive direction. There are distance marks on the axes
(for example $10$\thinspace km) to give some sense of scale. Below the top
border of the window, the azimuth and elevation angle of the current
viewpoint are shown. Using the mouse, one can manipulate the coordinate
system in three ways:

\begin{enumerate}
\item  Rotate: To rotate the axes, push and hold the left mouse button and
move the mouse in the required direction. Moving it up and down will change
the elevation angle, moving it left and right changes the azimuth angle.

\item  Zoom: To zoom in or out, push and hold the right mouse button and
move it up or down. The distance marks on the axes should start moving.

\item  There is a way of changing the focus point (the point around which
the coordinate system turns). This is described in Section~\ref{focuspoint}.
\end{enumerate}

In the top-left corner the current simulation time is given, i.e.~what one
sees is a snapshot at that given time. This helps visualising the position
of point targets and platforms at specific times. The time instance can be
changed by entering a time into the edit-box or by clicking on the
left/right scroll buttons just to the right of the edit-box.

On the left, a box containing a list of all objects is given. The order
depends on the relative positioning, i.e.~all objects placed on a platform
will appear (indented) just below it.

\subsection{Placing objects}

To place a new object (point target, radar or platform), select the required
object from the ``Geometry'' menu. A dialog with a whole set of parameters
will appear. All of the parameters are set to some default value and most of
them don't have to be changed for most of the simpler simulations. Use the 
\textbf{Tab} key or the mouse to navigate through the dialogs. By clicking
on the \textbf{OK} button, the object will appear on the screen (if in
sight). A short-cut to create new objects is the toolbar on top of the
screen (see Figure~\ref{toolbar}).

\FRAME{fhFU}{2.9533in}{0.2949in}{0pt}{\Qcb{The toolbar}}{\Qlb{toolbar}}{%
toolbar.gif}{\special{language "Scientific Word";type
"GRAPHIC";maintain-aspect-ratio TRUE;display "ICON";valid_file "F";width
2.9533in;height 0.2949in;depth 0pt;original-width 529.25pt;original-height
48.9375pt;cropleft "0";croptop "1";cropright "1";cropbottom "0";filename
'D:/SARSIM2/DOC/TEX/GRAPHICS/Toolbar.gif';file-properties "XNPEU";}}

\subsection{Editing objects}

There are two ways to edit objects. One is to double-click the object on the
list given on the left side, the other is to right-click the object itself
in the coordinate system. This only applies to point targets and
radars---use the list to modify platforms.

\subsection{Deleting objects}

To delete a point target or radar right-click it in the coordinate system
and choose ``Delete Point Target'' / ``Delete Radar'' from the appearing
menu. To delete a platform choose the required platform from the list.

\section{A simple example}

This section will show how to create a simple simulation with a single point
target. The simulation parameters are given in Table~\ref{tab:simparams}.

%TCIMACRO{\TeXButton{B}{\begin{table}[tbp] \centering}}
%BeginExpansion
\begin{table}[tbp] \centering%
%EndExpansion
\begin{tabular}{|c|c|}
\hline
Type of pulse modulation & chirp pulse \\ \hline
$B$ & 50 MHz \\ \hline
$f_{c}$ & 1 GHz \\ \hline
$T_{p}$ & 5000 ns \\ \hline
PRF & 1 kHz \\ \hline
$P_{tx}$ & 1 kW \\ \hline
antenna gain & isotropic \\ \hline
$d$ & 3000 m \\ \hline
$\sigma $ & 3 m$^{2}$ \\ \hline
\end{tabular}
\caption{Simulation Parameters\label{tab:simparams}}%
%TCIMACRO{\TeXButton{E}{\end{table}}}
%BeginExpansion
\end{table}%
%EndExpansion

\begin{itemize}
\item  Clear (if necessary) the current simulation by selecting \textit{File
/ New} on the menu.

\item  Create a new radar by selecting \textit{Geometry / New Radar} (or by
clicking on the \textbf{Radar} button). A window as shown in Figure~\ref
{exradar} will appear.\FRAME{fhFU}{3.4281in}{3.4151in}{0pt}{\Qcb{Radar setup
window (1/2)}}{\Qlb{exradar}}{exradar.gif}{\special{language "Scientific
Word";type "GRAPHIC";maintain-aspect-ratio TRUE;display "ICON";valid_file
"F";width 3.4281in;height 3.4151in;depth 0pt;original-width
0pt;original-height 0pt;cropleft "0";croptop "1";cropright "1";cropbottom
"0";filename 'D:/SARSIM2/DOC/TEX/GRAPHICS/Exradar.gif';file-properties
"XNPEU";}}

Change the pulse type to ``Chirp'' by selecting the \textbf{Chirp} radio
button (circled in Figure~\ref{exradar}). (Radio buttons are the round
circles with one having a black dot in it, meaning this option is selected).
The remaining parameters will have the correct parameters by default. Click
the \textbf{OK} button. A radar (dotted circle with a grey disc in it) will
appear at the origin of the \textbf{Earth} coordinate system. On the list at
the left a ``Radar: Radar1'' entry under ``Platform: Earth'' will appear.

\item  Create a point target by selecting \textit{Geometry / New Target} (or
the according \textbf{Target} button).
\end{itemize}

\FRAME{fhFU}{3.3088in}{2.3713in}{0pt}{\Qcb{Point target setup window}}{\Qlb{%
extarget}}{extarget.gif}{\special{language "Scientific Word";type
"GRAPHIC";maintain-aspect-ratio TRUE;display "ICON";valid_file "F";width
3.3088in;height 2.3713in;depth 0pt;original-width 0pt;original-height
0pt;cropleft "0";croptop "1";cropright "1";cropbottom "0";filename
'D:/SARSIM2/DOC/TEX/GRAPHICS/Extarget.gif';file-properties "XNPEU";}}

\begin{itemize}
\item  Set the $x$-coordinate to $3000$\thinspace m and the radar cross
section to $3$\thinspace m$^{2}$ as indicated in Figure~\ref{extarget}.
Click on \textbf{OK}. A point target (red dot with black circle around it)
will appear.

\item  Select \textit{Simulation / Raw return} from the menu. Another window
will open, shown in Figure~\ref{exrawret}.
\end{itemize}

\FRAME{fhFU}{3.5648in}{2.4405in}{0pt}{\Qcb{Return signal of a single point
target}}{\Qlb{exrawret}}{exrawret.gif}{\special{language "Scientific
Word";type "GRAPHIC";maintain-aspect-ratio TRUE;display "ICON";valid_file
"F";width 3.5648in;height 2.4405in;depth 0pt;original-width
0pt;original-height 0pt;cropleft "0";croptop "1";cropright "1";cropbottom
"0";filename 'D:/SARSIM2/DOC/TEX/GRAPHICS/Exrawret.gif';file-properties
"XNPEU";}}

Shown on the right-hand side of this window is a colour-coded image of the
received waveform displayed as slant range ($x$-direction) in meter versus
azimuth time ($y$-direction) in seconds. As the point target is stationary,
the position of the point target in slant range stays constantly at $3000$
meters. The colour palette has been mapped such that light blue represents
the highest positive value, black represents zero and light red the highest
negative value. The waveform is shown at baseband, i.e.~the frequency
spectrum has been shifted such that the carrier frequency is removed.
Initially the pulse shown does not resemble a chirp pulse, because aliasing
takes place on the screen, i.e.~it is necessary to zoom in, to see the
details of the chirp pulse.

On the left side the current radar, the simulation window (slant range /
azimuth) and the display type (real / imaginary / magnitude / phase) can be
chosen.

To select a different radar, click on the desired radar on the list. For
this simulation only a single radar is defined.

The simulation window size (slant range and azimuth) can be changed in four
ways:

\begin{enumerate}
\item  Changing the values of the slant/azimuth range in the \textbf{%
edit-boxes} on the left.

\item  Zooming in by selecting an area on the image with the left mouse
button. This is done by pushing and holding the left mouse button on the
desired corner and releasing it on the opposite corner. The window will
update automatically.

\item  Zooming out by a factor of two in either direction by right-clicking
anywhere on the image. The window will also update automatically.

\item  Showing all targets (i.e.~selecting a slant range spanning from the
closest to the furthest target) by clicking on the \textbf{Show all targets}
button.
\end{enumerate}

If the azimuth scale is such that two pulses are separate by at least $20$
pixels, the actual waveform of the pulse will be displayed as a graph as
shown in Figure~\ref{exzoom}.

\FRAME{fhFU}{4.5048in}{3.1713in}{0pt}{\Qcb{Raw return after zooming in}}{%
\Qlb{exzoom}}{exzoom.gif}{\special{language "Scientific Word";type
"GRAPHIC";maintain-aspect-ratio TRUE;display "ICON";valid_file "F";width
4.5048in;height 3.1713in;depth 0pt;original-width 0pt;original-height
0pt;cropleft "0";croptop "1";cropright "1";cropbottom "0";filename
'D:/SARSIM2/DOC/TEX/GRAPHICS/Exzoom.gif';file-properties "XNPEU";}}

The time instance at which the shown pulses has been transmitted by the
radar can then be easily identified, for example in this example the PRF is $%
1$\thinspace kHz, therefore a pulse will be transmitted every $0.001$
seconds. Note that the graphs of the pulses do not use the azimuth time
scale on the left, i.e.~the amplitude of the graphs do not correspond to the
scale in seconds given at the left, but they have their own amplitude scale
(not shown). Below the image, the highest absolute magnitude within the
displayed image is shown in millivolt. Note that this value might not be
accurate due to the sampling process (i.e.~the waveform might have been
sampled slightly off-peak). For the current simulation a value of $%
4.095\cdot 10^{-5}$\thinspace mV is shown. For this simple simulation the
radar equation reduces to:

\begin{eqnarray}
P_{rx} &=&\frac{P_{tx}\cdot G^{2}\cdot \lambda ^{2}\cdot \sigma }{(4\cdot
\pi )^{3}\cdot d^{4}} \\
&=&\frac{1000\cdot 1\cdot 0.3^{2}\cdot 3}{(4\cdot \pi )^{3}\cdot 3000^{4}} \\
&=&1.67976\cdot 10^{-15}\,\text{W}
\end{eqnarray}

Therefore the amplitude is given by: 
\begin{eqnarray}
Magnitude &=&\sqrt{P_{rx}} \\
&=&4.095\cdot 10^{-5}\,\text{mV}
\end{eqnarray}

The whole window can be resized (or maximised) in the usual way, but note
that the calculation time is proportional to the image size. On slow
computers it might be advisable to make the window smaller than the default
size.

To \textbf{save} the window displayed, click on the ``Save Data'' button.
This function will be explained in more detail in Section~\ref{savedata}.
