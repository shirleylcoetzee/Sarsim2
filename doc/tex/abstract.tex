%TCIDATA{LaTeXparent=0,0,thesis.tex}
                      

\chapter*{Abstract}

\addcontentsline{toc}{chapter}{Abstract}

This dissertation describes the design and implementation of a radar
simulator called Sarsim2. The radar simulator was originally developed to
produce \textit{synthetic range profiles} (SRPs) of complex aircraft models.
It was then expanded and upgraded to generate simulated \textit{synthetic
aperture radar} (SAR) data.

Over the last few years a substantial amount of work has been carried out by
the \textit{Radar Remote Sensing Group} (RRSG) at the \textit{University of
Cape Town }(UCT) to produce SRPs of aircraft targets using an L-Band search
radar of Reutech Systems. The high range resolution that can be obtained
from SRP processing makes it feasible to extract characteristic features
from a profile obtained from an aircraft. The ultimate aim of producing SRPs
is to use these extracted features for\textit{\ non-cooperative target
recognition} (NCTR), i.e.\ to be able to identify an aircraft type from the
echo signal received by the radar. The radar simulator was written to
produce SRPs of aircraft models, which could then be used to investigate the
feasibility of various aircraft-identification algorithms.

The stepped-frequency processing required to obtain SRPs of aircraft targets
has initiated further research in the RRSG into more efficient
stepped-frequency processing techniques, and the radar simulator has been
used extensively to generate simulated data.

The RRSG group is also actively involved with SAR processing techniques, and
the radar simulator has been invaluable in providing necessary simulation
data to test various processing algorithms.

One of the main objectives of this simulator was to have an easy-to-use
graphical interface, which can show results in real-time. This requirement
makes it necessary to find some way of reducing the required computation.
The solution implemented may be called WYSIWIC (what you see is what is
calculated). This means that the data is only calculated to a resolution
depending on the screen resolution. Only when the data is saved to disk will
it be calculated and written with the required sampling rate.

Some of the features of the radar simulator include:

\begin{itemize}
\item  Chirp, monochrome and user-defined pulse modulations

\item  Stepped-frequency implementation with constant or user-defined
frequency increments

\item  Independent moving platforms with user-defined paths

\item  Generation of text script files

\item  Configurable A/D conversion

\item  Angle dependent \textit{radar cross section} (RCS) of point targets

\item  Rotating antennas, spot mode SAR

\item  Point target and platform motion errors

\item  All user-defined functions can be imported by a separate text file

\item  Powerful image viewer which can display SAR files of practically any
size
\end{itemize}

The program has been kept flexible so that features (for example using
real-life antenna gain patterns) can be loaded with ease.

The graphical frontend of the simulator was initially written in C++ using
the \textit{Object Windows Library }(OWL) of Borland C++~5.0, but was then
rewritten with Borland C++ Builder, which made the development much easier.
Borland C++ Builder (BCB) can be considered as the best \textit{Rapid
Application Development }(RAD) tool currently available, offering visual
components combined with the flexibility and speed of C++. The simulator
exploits the protected memory model of 32 bit programs which has the
advantages of crash protection and practically no memory restrictions. It
therefore has to run under either Windows~95 or Windows~NT~4.0. For
portability reasons all the code (except the windows front end) has been
written in ANSI~C++. All processor intensive calculation routines have been
written as threads. This makes other tasks running in the background more
responsive and also enables the user to abort a calculation at any time.

A second program has been included, which is basically the same program
without the graphical frontend. This program is portable as it is written in
pure ANSI~C++. It performs like a compiler which reads the script files
(text files) and writes the required simulation files to disk.
