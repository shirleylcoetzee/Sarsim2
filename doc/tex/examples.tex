%TCIDATA{LaTeXparent=0,0,THESIS.TEX}
                      

\chapter{Example Files\label{chapter:examples}}

Various examples for different applications will be described in this
chapter. The script files for all examples are available in the \textbf{%
EXAMPLES} subdirectory.

\section{A typical C-band SAR Application}

For this SAR application a plane is flying in a straight line at some height
over an area, which has to be investigated. The radar is mounted on the
plane with a sideways looking beam. The simulation file can be found as ``%
\textbf{sar\_r.scr'}' in the ``\textbf{examples'}' directory.

This example will model the following scenario:

A plane is flying at $200$\thinspace m/s parallel to the $y$-axis at a
height of $2000$ meters and $10$\thinspace km horizontal distance from the
area of interest as shown in Figure~\ref{cbandplane}.

\FRAME{fhFU}{2.1127in}{1.7132in}{0pt}{\Qcb{Geometry setup}}{\Qlb{cbandplane}%
}{cbandplane.eps}{\special{language "Scientific Word";type
"GRAPHIC";maintain-aspect-ratio TRUE;display "ICON";valid_file "F";width
2.1127in;height 1.7132in;depth 0pt;original-width 0pt;original-height
0pt;cropleft "0";croptop "1";cropright "1";cropbottom "0";filename
'D:/SARSIM2/DOC/TEX/GRAPHICS/CBANDPLANE.EPS';file-properties "XNPEU";}}

A sidelooking radar is mounted on the plane, the depression angle of the
beam is $11.3$ degrees. There have been $24$ point targets placed in an
``R'' shape, extending over $200$\thinspace x\thinspace $400$\thinspace m as
shown in Figure~\ref{r_setup}.

\FRAME{fhFU}{4.0404in}{3.902in}{0pt}{\Qcb{Point target setup}}{\Qlb{r_setup}%
}{r_setup.gif}{\special{language "Scientific Word";type
"GRAPHIC";maintain-aspect-ratio TRUE;display "ICON";valid_file "F";width
4.0404in;height 3.902in;depth 0pt;original-width 920.6875pt;original-height
889.0625pt;cropleft "0";croptop "1";cropright "1";cropbottom "0";filename
'D:/SARSIM2/DOC/TEX/GRAPHICS/R_setup.gif';file-properties "XNPEU";}}

The radar parameters are given in Table~\ref{tab:three}.

%TCIMACRO{\TeXButton{B}{\begin{table}[tbp] \centering}}
%BeginExpansion
\begin{table}[tbp] \centering%
%EndExpansion
\begin{tabular}{|l|l|}
\hline
pulse waveform & chirp pulse \\ \hline
$B$ & 100 MHz \\ \hline
$f_{c}$ & 5.3 GHz \\ \hline
$T_{p}$ & 800 ns \\ \hline
PRF & 300 Hz \\ \hline
$P_{tx}$ & 1 kW \\ \hline
losses & none \\ \hline
noise & none \\ \hline
beamwidth & 30 degrees \\ \hline
\end{tabular}
\caption{Radar Parameters\label{tab:three}}%
%TCIMACRO{\TeXButton{E}{\end{table}}}
%BeginExpansion
\end{table}%
%EndExpansion

First the PLANE platform is created. This is done by creating a new platform
with the name ``PLANE'' and defining a trajectory for it. The $x$-coordinate
decreases linearly by $200$\thinspace m/s, the $y$-coordinate stays constant
at $-10000$\thinspace m and the $z$-coordinate stays constant at $2000$%
\thinspace m. Therefore we define the following:

\begin{itemize}
\item  Position $x=100$\thinspace m at time = $0$\thinspace s

\item  Position $x=-100$\thinspace m at time = $1$\thinspace s

\item  Position $y=-10000\,$m at time = $0$\thinspace s (one point suffices)

\item  Position $z=2000$\thinspace m at time = $0$\thinspace s (one point
suffices)
\end{itemize}

The plane will move $200$ meters in $1$ second, being closest to the earth
origin at $0.5$ seconds.

The next step is to create the radar on the PLANE platform, the given
parameters need to be entered in the dialogs.

The last step would be to create the point target setup as shown in Figure~%
\ref{r_setup}. For this setup a simulation has been stored already and can
be recalled by selecting \textbf{Simulation} / \textbf{Previous}. The window
shown in Figure~\ref{cbandsim} will appear.

\FRAME{fhFU}{3.9678in}{3.3088in}{0pt}{\Qcb{Simulation window}}{\Qlb{cbandsim}%
}{cbandsim.gif}{\special{language "Scientific Word";type
"GRAPHIC";maintain-aspect-ratio TRUE;display "ICON";valid_file "F";width
3.9678in;height 3.3088in;depth 0pt;original-width 0pt;original-height
0pt;cropleft "0";croptop "1";cropright "1";cropbottom "0";filename
'D:/SARSIM2/DOC/TEX/GRAPHICS/Cbandsim.gif';file-properties "XNPEU";}}

Because the return signals of the point targets interfere with each other,
the original R-shape cannot be seen. However after saving this simulation
window and performing range and azimuth compression on it (in this specific
example the Chirp Scaling Algorithm has been applied), the image can be
restored as shown in Figure~\ref{sar_r}. Some artifacts appear on the upper
and lower border due to the fact that the azimuth range should have covered
more time.

\FRAME{fbFU}{2.1776in}{2.5996in}{0pt}{\Qcb{Image after processing}}{\Qlb{%
sar_r}}{sar_r.bmp}{\special{language "Scientific Word";type
"GRAPHIC";maintain-aspect-ratio TRUE;display "ICON";valid_file "F";width
2.1776in;height 2.5996in;depth 0pt;original-width 306.6875pt;original-height
366.8125pt;cropleft "0";croptop "1";cropright "1";cropbottom "0";filename
'D:/SARSIM2/DOC/TEX/GRAPHICS/Sar_r.bmp';file-properties "XNPEU";}}

The simulation script file will look as shown in Figure~\ref{cbandsimscr}.

\FRAME{fhFU}{6.1376in}{4.4685in}{0pt}{\Qcb{Shortened script file for C-band
SAR example }}{\Qlb{cbandsimscr}}{cbandsimscr.eps}{\special{language
"Scientific Word";type "GRAPHIC";maintain-aspect-ratio TRUE;display
"ICON";valid_file "F";width 6.1376in;height 4.4685in;depth
0pt;original-width 361.375pt;original-height 266.375pt;cropleft "0";croptop
"1";cropright "1";cropbottom "0";filename
'D:/SARSIM2/DOC/TEX/GRAPHICS/CBANDSIMSCR.EPS';file-properties "XNPEU";}}
